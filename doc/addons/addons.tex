
\documentclass[12pt,a4paper]{article}
\usepackage[latin1]{inputenc}
\usepackage{graphicx}
\usepackage{listings}

\lstset{language=lisp,escapechar=/}

\begin{document}

\section{Zugriff auf Analogports}

\begin{itemize}
\item
  \begin{lstlisting}
    (port-on /\emph{int}/)
  \end{lstlisting}
  Setzt den Port mit Nummer \emph{int} auf aktiv.
\item
  \begin{lstlisting}
    (port-off /\emph{int}/)
  \end{lstlisting}
  Setzt den Port mit Nummer \emph{int} auf passiv.
\item
  \begin{lstlisting}
    (port /\emph{int}/)
  \end{lstlisting}
  Liest den Port mit nummer \emph{int} aus.
\end{itemize}

\section{Kommunikation RCX -- RCX}

\begin{itemize}
\item
  \begin{lstlisting}
    (set-receive-mode /\emph{int}/)
  \end{lstlisting}
  Setzt den Empfangsmodus und gibt den alten Empfangsmodus zur\"uck.
  Dabei sind zwei verschiedene Modi m\"oglich:
  \begin{itemize}
  \item \verb+:lock+ (Wert 0): Ein Paket im Empfangsbuffer wird nicht
    \"uberschrieben. Neu empfangene Pakete werden so lange
    verworfen bis der Empfangspuffer ausgelesen wurde (\emph{receive}) oder
    entsperrt wurde (\emph{unlock-receive-buffer})
  \item \verb+:overwrite+ (Wert 1):.Ein neues Paket \"uberschreibt
    alte Pakete im Empfangsbuffer immer.
  \end{itemize}
\item
  \begin{lstlisting}
    (get-receive-mode)
  \end{lstlisting}
  Gibt den aktuellen Empfangsmodus zur\"uck.
\item
  \begin{lstlisting}
    (unlock-receive-buffer)
  \end{lstlisting}
  Gibt den Empfangsbuffer frei, damit neue Pakete im Empfangsmodus \verb+:lock+
  empfangen werden k\"onnen.
\item
  \begin{lstlisting}
    (transmit /\emph{int}/ /\emph{list}/)
  \end{lstlisting}
  \"Ubertr\"agt die Liste \emph{list} an den RCX mit der Adresse \emph{int}.
  \emph{list} darf keine verschachtelte Liste sein.
\item
  \begin{lstlisting}
    (input?)
  \end{lstlisting}
  Liefert \verb+#t+ zur\"uck, wenn ein Paket im Empfangsbuffer liegt, ansonsten
  \verb+#f+.
\item
  \begin{lstlisting}
    (receive)
  \end{lstlisting}
  Wartet auf ein Paket im Empfangsbuffer (falls nicht bereits ein
  Paket vorliegt) und liefert die empfangene Liste zur\"uck. Der
  Empfangsbuffer wird dabei automatisch freigegeben.
\item
  \begin{lstlisting}
    (set-lnp-mode /\emph{int}/)
  \end{lstlisting}
  Setzt den Kommunikationsmodus auf den angegebenen wert. Folgende
  zwei Modi sind m\"oglich:
  \begin{itemize}
  \item \verb+:far+ (Wert 1): Konfiguration f\"ur gro�e Reichweite.
  \item \verb+:near+ (Wert 0): Konfiguration f\"ur niedrige
    Reichweite.
  \end{itemize}
  Der R\"uckgabewert ist der vorherige Kommunikationsmodus.
\item
  \begin{lstlisting}
    (lnp-far-mode?)
  \end{lstlisting}
  Liefert \verb+#t+ zur\"uck, wenn der aktuelle Kommunikationsmodus
  \verb+:far+ ist, ansonsten \verb+#f+.
\item
  \begin{lstlisting}
    (set-host-addr /\emph{int}/)
  \end{lstlisting}
  Setzt die Hostadresse, unter der der Brick erreichbar ist.
\item
  \begin{lstlisting}
    (get-host-addr)
  \end{lstlisting}
  Liefert die aktuelle Hostadresse.
\end{itemize}
\end{document}
